%\documentclass[border={0pt 20pt 0pt 0pt}]{standalone}

\documentclass[border=1cm]{standalone}

\usepackage{tikz}
\usepackage{circuitikz}
\usepackage{adjustbox}
\usepackage{verbatim}

\pagenumbering{gobble}



\makeatletter
% used to process styles for to-path
\def\TikzBipolePath#1#2{\pgf@circ@bipole@path{#1}{#2}}
% restore size value for bipole definitions
\pgf@circ@Rlen = \pgfkeysvalueof{/tikz/circuitikz/bipoles/length}
\makeatother
\newlength{\ResUp}
\newlength{\ResDown}
\newlength{\ResLeft}
\newlength{\ResRight}

%  josephsonjunction
\ctikzset{bipoles/JJ/height/.initial=.30}   % box height
\ctikzset{bipoles/JJ/width/.initial=.30}    % box width
\pgfcircdeclarebipole{}                                    % no extra anchors
{\ctikzvalof{bipoles/JJ/height}}
{JJ}                                        % component name
{\ctikzvalof{bipoles/JJ/height}}
{\ctikzvalof{bipoles/JJ/width}}
{                                                          % component symbol drawing...
  \pgfextracty{\ResUp}{\northeast}                         % coordinates
  \pgfextracty{\ResDown}{\southwest}
  \pgfextractx{\ResLeft}{\southwest}
  \pgfextractx{\ResRight}{\northeast}
  \pgfsetlinewidth{3\pgfstartlinewidth}
  \pgfmoveto{\pgfpoint{\ResLeft}{\ResDown}}
  \pgflineto{\pgfpoint{\ResRight}{\ResUp}}
  \pgfmoveto{\pgfpoint{\ResRight}{\ResDown}}
  \pgflineto{\pgfpoint{\ResLeft}{\ResUp}}
  \pgfusepath{draw}
  \pgfsetlinewidth{\pgfstartlinewidth}
  \pgfmoveto{\pgfpoint{\ResLeft}{0}}
  \pgflineto{\pgfpoint{\ResRight}{0}}
  \pgfusepath{draw}
}
\def\circlepath#1{\TikzBipolePath{JJ}{#1}}
\tikzset{JJ/.style = {\circuitikzbasekey, /tikz/to path=\circlepath, l=#1}}




\begin{document}

\begin{comment}
\begin{adjustbox}{scale=1}
  \begin{circuitikz}
    \draw (0,0) 
    to[JJ, v^>=$\varphi_5$] (3,5)
    to[L, v^>=$\varphi_4$] (6,0)
    to[C, v^>=$\varphi_6$] (0,0);
    \draw (3,2)
    to[L, v^>=$\varphi_2$] (0,0);
    \draw (3,2)
    to[C, v_>=$\varphi_1$] (3,5);
    \draw (3,2)
    to[JJ, v_>=$\varphi_3$] (6,0);
  \end{circuitikz}
\end{adjustbox}
\end{comment}


\begin{adjustbox}{scale=1}
  \begin{circuitikz}
    \draw (0,0) 
    node[label={below:$\phi_1$}] {}
    to[JJ, *-*] (3,5)
    node[label={above:$\phi_3$}] {}
    to[L, *-*] (6,0)
    node[label={below:$\phi_2$}] {}
    to[C] (0,0);
    \draw (3,2)
    node[label={below:$\phi_0$}] {}
    to[L] (0,0);
    \draw (3,2)
    to[C] (3,5);
    \draw (3,2)
    to[JJ] (6,0);
  \end{circuitikz}
\end{adjustbox}


\end{document}